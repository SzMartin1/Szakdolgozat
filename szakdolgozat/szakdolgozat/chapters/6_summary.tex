\Chapter{Összefoglalás}

A szakdolgozatomban bemutattam a platformer játékok fejlődését, valamint ezeknek a játékoknak a procedurális mapgeneráláshoz való kapcsolódását. Sikeresen bemutattam a különböző játékmotorok néhány tulajdonságát, a Unity-ben használt mechanizmusok, koncepciók mélyebb részletezésével együtt. 

Láthattunk három darab olyan algoritmust, aminek a segítségével automatikusan lehet pályát generálni egy 2D-s játékhoz. Ez a három algoritmus a Perlin-zaj, a celluláris automata valamint a véletlen bolyongás. Én a véletlen bolyongás algoritmus választottam a játékomhoz, mivel ez illett az általam létrehozott preferenciákhoz a legjobban. Sajnos azt nem sikerült elérnem, hogy a játékos preferenciáit figyelembe véve generáljon pályát, de ezt orvosolni fogom egy saját menü létrehozásával. A jövőben szeretném vegyíteni a pályageneráló algoritmusokat, valamint nem csak 2D-s, hanem 3D-s játékhoz is alkalmaznám ezeket.

A szakdolgozatom hozzájárult a videójáták fejlesztés megismeréséhez is, hiszen láthattuk, hogyan lehet \texttt{GameObject}-ekből valamint a hozzájuk csatolható komponensekből egy teljesen működő 2D-s platformer játékot létrehozni. Részleteztem a fejlesztés során létrehozott osztályokat, valamint metódusokat és igyekeztem ezeket minél érthetőbben elmagyarázni, azonban néhol túlságosan is implementáció-közeli lett a dolgozatom.

A játék sajnálatos módon nincs még olyan állapotban, hogy a kereskedelmi forgalomban is megállja a helyét. Hiányoznak még belőle a megvizsgálható tárgyak, több szintet kellene létrehoznom a játékban, valamint több játszható karakter is növelné a játék élményét, játszhatóságát.

Összességében elégedett vagyok magammal, hogy sikerült elkészítenem életem első játékát, valamint még az automatikus pályagenerálást is sikerült beleépítenem a játékomba. A szerzett tudást mindenképpen fel fogom használni a jövőben, mondjuk egy 3D-s játék készítésénél.