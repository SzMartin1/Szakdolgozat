\Chapter{Bevezetés}

A modern számítógépes játékpiac rohamos fejlődése mellet a játékfejlesztők folyamatosan keresik azokat az innovatív megoldásokat, amelyekkel újszerű és lenyűgöző játékélményt hozhatnak létre. Az oldalnézetes (side-scroller) platformer játékok, amelyek az elmúlt évtizedekben jelentős népszerűségre tettek szert, különleges teret biztosítanak a kreativitás és a technológia találkozásának. Azonban ezeknek a játékoknak a fejlesztése komplex kihívásokkal jár, különösen a pályatervezés tekintetében, ahol a fejlesztőknek egyensúlyt kell találniuk az innováció, a játékélmény és a fejlesztési erőforrások között.

A szakdolgozatomban kettős célt tűztem ki: egyrészt egy teljesen működő, játszható 2D platformer játék tervezése és implementálása a Unity keretrendszerben, C\# programozási nyelven, másrészt egy hozzá kapcsolódó pályagenerátor algoritmus fejlesztése, amely képes automatikusan, a felhasználó preferenciáit alapul véve változatos és kihívást jelentő pályákat létrehozni. Ez a kettős megközelítés lehetővé teszi, hogy nem csak elméleti síkon vizsgáljuk a pályageneráló algoritmusokat, hanem valós játékkörnyezetben is teszteljük azok hatékonyságát és hatását a játékélményre.

A szakdolgozatom kiterjed a platformer játék fejlesztésének minden aspektusára, beleértve a játékmechanika megtervezését, a grafikai elemek integrálását, valamint a felhasználói interfész megvalósítását. Mindezek mellett a fő hangsúly a pályagenerátor algoritmuson van, amely a játék alapvető részét képezi. Az algoritmus tervezésekor különös figyelmet fordítok a paraméterezhetőségre és az adaptivitásra, hogy a generált pályák ne csak változatosak és kihívást jelentőek legyenek, hanem jól illeszkedjenek a játék dinamikájához és stílusához.

A szakdolgozat során a platformer játék fejlesztési folyamatának minden lépését alaposan dokumentálom, a kezdeti koncepciótól a végleges implementációig. Ezen túlmenően, az algoritmus tervezése és implementációja során részletesen bemutatom a különböző programozási kihívásokat, a paraméterezési stratégiákat, és azokat a tesztelési módszereket, amelyekkel az algoritmus teljesítménye és a generált pályák játékbeli hatékonysága értékelésre kerül.
A szakdolgozatom így nem csak egy konkért algoritmus kidolgozására vállalkozik, hanem hozzájárul a videójáték fejlesztés megismeréséhez is.

